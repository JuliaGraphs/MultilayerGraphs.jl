% This is abstarct.tex, a sample chapter demonstrating the
% LLNCS macro package for Springer Computer Science proceedings;
% Version 2.21 of 2022/01/12
%
\documentclass[runningheads]{llncs}
%
\usepackage[T1]{fontenc}
% T1 fonts will be used to generate the final print and online PDFs,
% so please use T1 fonts in your manuscript whenever possible.
% Other font encodings may result in incorrect characters.
%
\usepackage{graphicx}
% Used for displaying a sample figure. If possible, figure files should
% be included in EPS format.
%
% If you use the hyperref package, please uncomment the following two lines
% to display URLs in blue roman font according to Springer's eBook style:
%\usepackage{color}
%\renewcommand\UrlFont{\color{blue}\rmfamily}
%
\begin{document}
%
\title{Multilayer Network Science in Julia \\ with MultilayerGraphs.jl}
%
%\titlerunning{Abbreviated paper title}
% If the paper title is too long for the running head, you can set
% an abbreviated paper title here
%
\author{
    Claudio Moroni\inst{1,2}\orcidID{0000-0003-1274-6937} \\ \and
    Pietro Monticone\inst{1,2}\orcidID{0000-0002-2731-9623}
}
%
\authorrunning{Moroni and Monticone}
% First names are abbreviated in the running head.
% If there are more than two authors, 'et al.' is used.
%
\institute{
University of Turin, Italy (\url{https://www.unito.it}) \\ \and
Interdisciplinary Physics Team, Italy (\url{https://github.com/InPhyT})
}
%
\maketitle % typeset the header of the contribution
%
% The abstract should briefly summarize the contents of the paper in
% 150--250 words.
\begin{abstract}
    
MultilayerGraphs.jl is a Julia package for the creation, manipulation and analysis 
of the structure, dynamics and functions of multilayer graphs. \\

A multilayer graph consists of multiple subgraphs called layers which 
can be interconnected through bipartite graphs called interlayers composed 
of the sets of vertices of two different layers and the edges between them. \\

In order to formally represent multilayer networks, multiple theoretical paradigms 
have been proposed and adopted to model a wide spectrum of high-dimensional, 
multi-scale, time-dependent complex systems including molecular,
neuronal, social, ecological and economic networks. \\

The package features an implementation that maps a standard integer-labelled 
vertex representation to a more user-friendly framework exporting all the objects 
a practitioner would expect such as nodes, vertices, layers, interlayers, etc. \\

MultilayerGraphs.jl has been integrated within the JuliaGraphs\footnote{https://github.com/JuliaGraphs.} 
and the JuliaDynamics\footnote{https://github.com/JuliaDynamics.} ecosystems through: 
\begin{itemize}
    \item the extension of Graphs.jl\footnote{https://github.com/JuliaGraphs/Graphs.jl.} with several methods and metrics including 
    the multilayer eigenvector centrality, the multilayer modularity and the Von Newman entropy; 
    \item the compatibility with Agents.jl\footnote{https://github.com/JuliaDynamics/Agents.jl.} allowing for agent-based modelling 
    on general multilayer networks. \\
\end{itemize}

For a comprehensive exploration of the package features and functionalities the reader 
is invited to consult the README\footnote{https://github.com/JuliaGraphs/MultilayerGraphs.jl/blob/main/README.md.} 
and documentation\footnote{https://juliagraphs.org/MultilayerGraphs.jl.}.

\keywords{
    Discrete Mathematics \and Graph Theory \and Network Science \and \\ 
    Multilayer Graphs \and Multilayer Networks \and Complex Systems \and \\
    Computer Science \and Julia Language.
    }

\end{abstract}
%
%
%
\section{First Section}
\subsection{A Subsection Sample}
Please note that the first paragraph of a section or subsection is
not indented. The first paragraph that follows a table, figure,
equation etc. does not need an indent, either.

Subsequent paragraphs, however, are indented.

\subsubsection{Sample Heading (Third Level)} Only two levels of
headings should be numbered. Lower level headings remain unnumbered;
they are formatted as run-in headings.

\paragraph{Sample Heading (Fourth Level)}
The contribution should contain no more than four levels of
headings. Table~\ref{tab1} gives a summary of all heading levels.

\begin{table}
\caption{Table captions should be placed above the
tables.}\label{tab1}
\begin{tabular}{|l|l|l|}
\hline
Heading level &  Example & Font size and style\\
\hline
Title (centered) &  {\Large\bfseries Lecture Notes} & 14 point, bold\\
1st-level heading &  {\large\bfseries 1 Introduction} & 12 point, bold\\
2nd-level heading & {\bfseries 2.1 Printing Area} & 10 point, bold\\
3rd-level heading & {\bfseries Run-in Heading in Bold.} Text follows & 10 point, bold\\
4th-level heading & {\itshape Lowest Level Heading.} Text follows & 10 point, italic\\
\hline
\end{tabular}
\end{table}


\noindent Displayed equations are centered and set on a separate
line.
\begin{equation}
x + y = z
\end{equation}
Please try to avoid rasterized images for line-art diagrams and
schemas. Whenever possible, use vector graphics instead (see
Fig.~\ref{fig1}).

\begin{figure}
\includegraphics[width=\textwidth]{fig1.eps}
\caption{A figure caption is always placed below the illustration.
Please note that short captions are centered, while long ones are
justified by the macro package automatically.} \label{fig1}
\end{figure}

\begin{theorem}
This is a sample theorem. The run-in heading is set in bold, while
the following text appears in italics. Definitions, lemmas,
propositions, and corollaries are styled the same way.
\end{theorem}
%
% the environments 'definition', 'lemma', 'proposition', 'corollary',
% 'remark', and 'example' are defined in the LLNCS documentclass as well.
%
\begin{proof}
Proofs, examples, and remarks have the initial word in italics,
while the following text appears in normal font.
\end{proof}
For citations of references, we prefer the use of square brackets
and consecutive numbers. Citations using labels or the author/year
convention are also acceptable. 

\cite{DeDomenico2013,Kivela2014,Bianconi2018}

\subsubsection{Acknowledgements} 

Write here something about JuliaLang, JuliaGraphs, etc.

This open-source research software project received no financial support.


% ---- Bibliography ----
%
% BibTeX users should specify bibliography style 'splncs04'.
% References will then be sorted and formatted in the correct style.
%
\bibliographystyle{splncs04}
\bibliography{paper}
\end{document}
